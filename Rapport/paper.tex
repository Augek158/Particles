\documentclass[journal]{vgtc}                % final (journal style)
%\documentclass[review,journal]{vgtc}         % review (journal style)
%\documentclass[widereview]{vgtc}             % wide-spaced review
%\documentclass[preprint,journal]{vgtc}       % preprint (journal style)
%\documentclass[electronic,journal]{vgtc}     % electronic version, journal

%% Uncomment one of the lines above depending on where your paper is
%% in the conference process. ``review'' and ``widereview'' are for review
%% submission, ``preprint'' is for pre-publication, and the final version
%% doesn't use a specific qualifier. Further, ``electronic'' includes
%% hyperreferences for more convenient online viewing.

%% Please use one of the ``review'' options in combination with the
%% assigned online id (see below) ONLY if your paper uses a double blind
%% review process. Some conferences, like IEEE Vis and InfoVis, have NOT
%% in the past.

%% Please note that the use of figures other than the optional teaser is not permitted on the first page
%% of the journal version.  Figures should begin on the second page and be
%% in CMYK or Grey scale format, otherwise, colour shifting may occur
%% during the printing process.  Papers submitted with figures other than the optional teaser on the
%% first page will be refused.

%% These three lines bring in essential packages: ``mathptmx'' for Type 1
%% typefaces, ``graphicx'' for inclusion of EPS figures. and ``times''
%% for proper handling of the times font family.

\usepackage{mathptmx}
\usepackage{graphicx}
\usepackage{multicol}
\usepackage{times}
\usepackage{balance}
\usepackage{fancyhdr}
\usepackage[nooneline,hang,it,IT]{subfigure}

%% We encourage the use of mathptmx for consistent usage of times font
%% throughout the proceedings. However, if you encounter conflicts
%% with other math-related packages, you may want to disable it.

%% This turns references into clickable hyperlinks.
\usepackage[bookmarks,backref=true,linkcolor=black]{hyperref} %,colorlinks
\hypersetup{
  pdfauthor = {},
  pdftitle = {},
  pdfsubject = {},
  pdfkeywords = {},
  colorlinks=true,
  linkcolor= black,
  citecolor= black,
  pageanchor=true,
  urlcolor = black,
  plainpages = false,
  linktocpage
}

%% If you are submitting a paper to a conference for review with a double
%% blind reviewing process, please replace the value ``0'' below with your
%% OnlineID. Otherwise, you may safely leave it at ``0''.
\onlineid{0}


%% declare the category of your paper, only shown in review mode
\vgtccategory{Research}

%% allow for this line if you want the electronic option to work properly
\vgtcinsertpkg

%% In preprint mode you may define your own headline.
%\preprinttext{To appear in an IEEE VGTC sponsored conference.}

%% Paper title.

%%\title{GPU Accelerated Particle System}

%% This is how authors are specified in the journal style

%% indicate IEEE Member or Student Member in form indicated below
\author{ \textbf{ Oscar Johnson} \\ \today}



%% Abstract section.
\abstract{
In this project facebook checkin data was plotted on Google maps. The objective was to make an intuitive visualisation of facebook checkins. Used in the project was Facebook Graph API, google maps and javascript. The result was a simple visualisation without data mining and with an additional statistics feature. The result corresponds to the original idea, however more focus on information visualization/data mining could be used for more interesting correlations on the data.
}
%% Keywords that describe your work. Will show as 'Index Terms' in journal
%% please capitalize first letter and insert punctuation after last keyword
\keywords{Facebook Graph API, Google Maps API.}


%% Paper begins


\begin{document}
	\begin{titlepage}
		\begin{center} 
			{\Huge GPU Accelerated Particle System} \\[1cm]

			{\LARGE Project in The Course TNCG14 Advanced Computer Graphics} \\[0.4cm]
			{\LARGE by August Ek and Oscar Johnson}
		\end{center}


	\end{titlepage}

	%% First page
	

	\section{Introduction}
The graphics processing unit have a lot of power which could be used in graphics calculations and even in arbitrary calculations surrounding these. One example well suited for this is a particle system. In a particle system these calculations could involve position, velocity, and lifetime of a particle, which all are not primarily graphical computations. These calculations could be done on the CPU, which is in comparison very inefficient. Forcing the CPU to call the GPU for each particle calculation, instead of once for all particles. 

After acceleration with the GPU a visualisation of the power gained was implemented. Using particles to demonstrate gravitational physics. And also with the use of perlin noise for simulation of smoke.

After acceleration with the GPU a visualisation of the power gained was implemented. Using particles to demonstrate gravitational physics. And also with the use of perlin noise for simulation of smoke.

	\section{Background and related work}
This project uses opengl 4.1 which is a constraint. No compute shader is available to make arbitrary calculations on the GPU. However a transform feedback is available which is very similar in functionality. The project also includes instancing, which basically creates a prototype of the particle and make all calculations on this particle, preventing for several calls to the GPU each time instance and fully make use of the GPU.

There are some related articles that we have used as material. These articles include information about smoke simulation using perlin noise.



	\section{Results}

	%%figure	

	\section{Implementation Details}	
	
	\subsection{Particles}

Each particle has two attributes: position and velocity. These two will need to be updated each frame before the particle is rendered to the screen. All of the particles are created at the start of the application and are then used to fill out the vertex buffer containing the positions/velocities a little by little.


	\subsection{Calculating positions}
The new positions for the particles was calculated using transform feedback. This means that a shader was written that had the single purpose to calculate new positions for each particle. The resulting buffer from the shader was then used to draw the particles on the screen. Using this method gives a huge performance boost since all of the calculations could be moved from the CPU to the GPU, utilizing the parallel process power of the GPU and not relying on the few cores offered by the CPU.  The figure below shows how the main particle loop works.





	%% Insert fig with caption


	\subsection{Rendering particles}
Instanced rendering was used to draw all of the particles with a single call to the GPU. This basically tells the GPU to render the same particle at different positions instead of render a different particle for each position. This offered a performance boost since the overhead related with making each API call would go from O(n) to 1. 




	\section{Simulation}
Two different simulations were made to test the capabilities of the particle system. The first one uses classical mechanics to demonstrate particles colliding with planes and the second uses a noise based velocity field to simulate smoke and fire. 


	\subsection{Classical mechanics}

The physics implementation includes basic gravitational force. For each frame a new velocity and position is calculated due to gravity. 
A bounding box is surrounding the particles allowing implementation of collision detection. This is done with a simple posteori collision detection. Meaning if the particle moves outside the boundings, move it within the boundings in and calculate the new velocity. The new velocity is then calculated with the reflection formula for vectors.

If collision with the ground level would happen, the new velocity gets scaled with a number between 0.0 and 1.0. This is the coefficent of restitution, which represents the energy loss of the particle upon impact.

A calculation of the particles energy is done each frame. Using the particles mass, the gravity, distance to the bottom of the bounding box and velocity. This yields the kinetic and potential energy of the particle. If the energy value is less than a small given value, the particle is assumed still. To avoid unnecessary computation and vibrations at small energy values the velocity is set to zero and the position is assign previews value from then on.


	\subsection{Smoke}
To make a smoke simulation correct from a physics viewpoint, there is the need to solve the equations of Navier-Stokes. However, to get a visually realistic result, procedural methods could be used to create the vector field which moves the particles. This is easier to implement, more memory efficient and gives a result that looks realistic enough for this implementation.

The velocity field used was generated by using classic perlin noise. The noise potential is simply a scalar field when implementing in 2D giving: 
	
	%%Formel 1

x, y, and z are the positions of the particle and time is the elapsed time since the simulation started.

Using the noise directly as the velocity field results in a field that diverges at certain areas thus producing a field with properties not desired for this type of simulation. Robert Bridson proposes a solution to this by using the curl of the velocity field since one classic calculus identity is that

	%%Formel 2

Calculating the curl is done in the way:
	%%Formel 3

Since only two dimensions were used, psi were set to:

	%%Formel 4

leading to:
	%% Formel 5

The partial derivatives is evaluated using simple finite differences with very small displacements. 

The result is a divergence-free field which is essential to produce realistic velocity fields such  as those of smoke or water.  A comparison of the two noises can be seen below.

	%% Two figures with caption


\section{Conclusion}
The goal to implement a basic particle simulator was reached and by some extent exceeded. The simulator is using recent GPU techniques to boost the performance and using physics and vector calculus to simulate realistic scenarios. Transform feedback combined with instanced rendering removes the bottleneck of being limited to a low amount of particles. The system can easily handle hundreds of thousands of particles at smooth FPS rates even on low-end hardware. The particle system offers high modularity due to the fact that it’s very simple to replace the calculating vertex shader to give the particles a totally different pattern of motion thus simulating different phenomena.

\section{Future work and Improvements}
The smoke simulation could be improved in the way that the smoke respected boundaries giving the visual result that the smoke avoids some obstacles. This requires that the distance gradient of all obstacles is mixed with the noise-potential and we felt that we didn’t have the time for this feature. 

Endless tweaking on parameters for the vector field could be done to create other visuals such as water. There is no real limit on how this could be done since the only goal is to create something aesthetically pleasant. Most of the work of this project had something to do with the motion of the particles and thus the calculations done in the vertex- and transform shader. A lot of work could be done in the fragment shader to drastically improve the visuals.
 
We can’t say that it wouldn’t be interesting to implement a physically accurate simulation with regards on the equations of Navier-Stokes. That simulation could then be used as reference to compare with the current one in aspects of performance and visual realism. 

A graphical user interface could be implemented to easy swap between different simulations and the user could also be able to change some of the parameters such as: number of particles in the system, size of each particle and the location where the particles should be spawned. 

\section{References}
	

	%% Hardware used.
\def\arraystretch{1.5}%
	\begin{table}[h]
		\begin{tabular}{| ll |}
			\hline
			\textbf{Macbook Air 2013} & \textbf{Macbook Pro 2012} \\[0.75ex] %%\hline
			 intel i5 1.3 GHz  &   2,5 GHz Intel Core i5  \\[0.75ex]
			  8GB RAM & 8GB RAM    \\[0.75ex]
		 	 Intel HD Graphics 5000 &  Intel HD Graphics 4000   \\ [0.75ex]
			\hline
		\end{tabular}
	\end{table}	

	%%Benchmark
	

	\begin{table}[h]
		\begin{tabular}{l | r | r}
			  & 30 FPS & 60 FPS   \\[0.75ex] \hline
  			No Instancing, no Transform Feedback & 10 000 & 5000  \\[0.75ex]
  			Instancing & 300 000 & 100 000  \\[0.75ex]
			Instancing and Transform Feedback & 4 000 000 & 1 000 000 \\[0.75ex]


		\end{tabular}
	\end{table}	

\end{document}
